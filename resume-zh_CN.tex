% ============================================================================
% 简历内容文件
% ============================================================================
\documentclass{resume} % 使用自定义的 resume 类

% ============================================================================
% 本文档所需包 
% ============================================================================
% \usepackage{zh_CN-Adobefonts_external} % 稍后添加

% ============================================================================
% 文档开始
% ============================================================================
\begin{document}
\pagenumbering{gobble}

% ============================================================================
% 个人信息部分
% ============================================================================
% --- 先定义联系信息 ---
\contactInfo
{139-xxxx-xxxx}{youremail@email.com}{username}{IELTS: 6.5}{男}

% --- 再定义姓名 ---
\name{姓~名}

% ============================================================================
% 教育背景部分
% ============================================================================
\section{\faGraduationCap\ 教育背景 (Education)} 
\noindent
\begin{tabular*}{\textwidth}{@{\extracolsep{\fill}} l l l r @{}}
    {\large \textbf{某某大学}} &
    {\textbf{某某专业}} &
    {\textit{学位与状态 (例如 硕士在读)}} &
    {\textbf{预计毕业日期 (例如 202X.XX)}} \\
\end{tabular*}

% ============================================================================
% 研究经历部分
% ============================================================================
\section{\faFlask\ 研究经历 (Research)} 
\datedsubsection{\textbf{研究项目标题}}
                 {关键技术栈 (例如 C++, Python)}
                 {所属机构 / 背景 (例如 某某研究院)}
\begin{itemize}
    \item 描述研究背景、担任的角色以及解决的问题。
    \item 详细说明使用或开发的方法、技术或算法。
    \item 突出关键发现、贡献或成果(例如:发表论文、开发原型、达成特定结果)。如有可能,量化成果。
\end{itemize}

% ============================================================================
% 实习与项目经历部分
% ============================================================================
\section{\faCogs\ 实习与项目经历 (Experience)} 
\datedsubsection{\textbf{公司名称 | 部门 (或项目)}}
                 {岗位名称 / 主要技术栈}
                 {202X.XX - 202X.XX}
\begin{itemize}
    \item 简要描述项目目标或团队目标。
    \item 详细说明具体职责和执行的任务。使用动作动词。
    \item 提及使用的工具、技术或方法。
    \item 尽可能量化工作成绩。
\end{itemize}

% ============================================================================
% 专业技能部分
% ============================================================================
\section{\faWrench\ 专业技能 (Skills)} % Icon added
\begin{itemize} % Note: parsep adjusted later
    \item \textbf{编程语言:} C++, Python, JavaScript, Java, ...
    \item \textbf{框架与库:} TensorFlow, PyTorch, React, ...
    \item \textbf{工具 (Tools):} Git, Docker, Linux, ...
\end{itemize}

% ============================================================================
% 荣誉奖项部分
% ============================================================================
\section{\faDiamond\ 荣誉奖项 (Honors & Awards)} 
\begin{itemize} 
  \item 示例奖项一 \hfill{获奖年份}
  \item 示例奖项二 \hfill{获奖年份/地点}
\end{itemize}

% ============================================================================
% 相关链接部分
% ============================================================================
\section{\faLink\ 相关链接 (Links)} 
\begin{itemize} 
    \item \textit{技术博客 (Technical Blog):} \href{https://your-blog-url.com}{your-blog-url.com}
    % \item \textit{GitHub 主页:} \href{https://github.com/username}{github.com/username}
\end{itemize}

% ============================================================================
% 自我评价部分
% ============================================================================
\section{\faThumbsOUp\ 自我评价 (Summary)} 
\begin{itemize} 
  \item 突出一个关键技术优势或专业领域。
  \item 提及优秀的软技能。
\end{itemize}

% ============================================================================
% 文档结束
% ============================================================================
\end{document}