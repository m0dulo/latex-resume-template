% ============================================================================
% 简历内容文件
% ============================================================================
\documentclass{resume} 

% ============================================================================
% 本文档所需包
% ============================================================================
\usepackage{zh_CN-Adobefonts_external} 
%\usepackage{zh_CN-Adobefonts_internal}
\usepackage{linespacing_fix} 

% ============================================================================
% 可选:设置照片
% ============================================================================
\photo{photo.png} % <--- 路径指向的照片文件,注释掉则无照片

% ============================================================================
% 文档开始
% ============================================================================
\begin{document}
\pagenumbering{gobble} % 不显示页码

% ============================================================================
% 个人信息部分
% ============================================================================
% --- 先定义联系信息 ---
\contactInfo
{139-1234-5678}{youremail@email.com}{username}{IELTS: 6.5}{男} 
% 参数: {电话/微信}{邮箱}{GitHub/网站}{英语}{性别} 

% --- 再定义姓名 ---
\name{姓~名} 

% --- 求职意向 (可选) ---
% \objective{目标职位}{实习时长 (可选)} 

% ============================================================================
% 教育背景部分 
% ============================================================================
\iconsection{\faGraduationCap}{教育背景}{EDUCATION} 
\noindent 
\begin{tabular*}{\textwidth}{@{\extracolsep{\fill}} l l l r @{}} 
    {\large \textbf{某某某大学}} & 
    {\textbf{某某专业}} & 
    {\textit{硕士研究生在读}} & 
    {\textbf{预计毕业日期 (例如 202X.XX)}} \\ 

    {\large \textbf{某某大学}} & 
    {\textbf{某某专业}} & 
    {\textit{工学学士}} & 
    {\textbf{毕业日期 (例如 202X.XX)}} \\ 
\end{tabular*}

% ============================================================================
% 研究经历部分
% ============================================================================
\iconsection{\faFlask}{研究经历}{RESEARCH}
\datedsubsection{\textbf{研究项目标题}} 
                  {关键技术栈 (例如 C++, Python)} 
                  {所属机构 / 背景 (例如 某某研究院)} 
\begin{itemize}
    \item 描述研究背景、担任的角色以及解决的问题。
    \item 详细说明使用或开发的方法、技术或算法。
    \item 突出关键发现、贡献或成果(例如:发表论文、开发原型、达成特定结果)。如有可能,量化成果。
\end{itemize}

% ============================================================================
% 实习与项目经历部分
% ============================================================================
\iconsection{\faCogs}{项目与实习经历}{EXPERIENCE}
\datedsubsection{\textbf{公司名称 | 部门 (或项目)}} 
                  {岗位名称 / 主要技术栈} 
                  {202X.XX - 202X.XX} 
\begin{itemize}
    \item 简要描述项目目标或团队目标。
    \item 详细说明具体职责和执行的任务。使用动作动词。
    \item 提及使用的工具、技术或方法。
    \item 尽可能量化工作成绩。
\end{itemize}

% ============================================================================
% 专业技能部分
% ============================================================================
\iconsection{\faWrench}{专业技能}{SKILLS}
\begin{itemize}[parsep=0.2ex] 
    \item \textbf{编程语言:} C++, Python, JavaScript, Java, Rust, Go, SQL, ...
    \item \textbf{框架与库:} TensorFlow, PyTorch, MLIR, LLVM, TensorRT, React, Vue, Spring Boot, ...
    \item \textbf{工具 (Tools):} Git, Docker, Kubernetes, CMake, Linux, Nsight, Perf, ...
\end{itemize}

% ============================================================================
% 荣誉奖项部分
% ============================================================================
\iconsection{\faDiamond}{荣誉奖项}{HONORS \& AWARDS} % 注意 & 需要写为 \&
\begin{itemize}[parsep=0.2ex]
  \item 示例奖项一 (例如:国家奖学金) \hfill{获奖年份} 
  \item 示例奖项二 (例如:某某竞赛一等奖) \hfill{获奖年份/地点} 
  \item 示例其他荣誉 (例如:知名开源项目贡献者) 
  \item 在此处继续添加其他条目...
\end{itemize}

% ============================================================================
% 相关链接部分
% ============================================================================
\iconsection{\faLink}{相关链接}{LINKS} 
\begin{itemize}[parsep=0.2ex]
    \item \textit{技术博客 (Technical Blog):} \href{https://your-blog-url.com}{your-blog-url.com} 
    \item \textit{项目作品集 (Portfolio):} \href{https://your-portfolio-url.com}{your-portfolio-url.com}
    \item \textit{其他相关链接或说明...} 
\end{itemize}

% ============================================================================
% 自我评价部分 
% ============================================================================
\iconsection{\faThumbsOUp}{自我评价}{SUMMARY}
\begin{itemize}[parsep=0.2ex]
  \item 突出一个关键技术优势或专业领域。
  \item 提及优秀的软技能,如团队合作、沟通或解决问题能力。
  \item 表达学习热情、对特定领域的兴趣或职业目标。
  \item 示例:学习能力强,具备优秀的分析能力,对某某领域充满热情。
\end{itemize}

% ============================================================================
% 文档结束
% ============================================================================
\end{document}